\begin{abstract}

Because of the significant Operation and Maintenance (O\&M) cost of wind farm, there have been growing interest in developing new maintenance strategy in the offshore wind industry. Condition-based maintenance (CBM) strategy, which base on the information provided by Condition Monitoring Systems (CMS), predicts component failures and help farm owner schedule maintenance in advance. However, the high cost of additional sensors installation, personnel and software leads to limited reliability and cost-effectiveness. On the other hand, Supervisory Control and Data Acquisition (SCADA) system, which is basic component of wind turbine, is installed in almost every large wind farm to record turbines' main parameters. For this reason, the possibility of analysing the information provided by SCADA system to detect abnormalities has gained more attention in recent years. The aim of this thesis is to investigate how machine learning algorithms can be applied to offshore wind turbine operational SCADA data to predict failures and estimate remaining useful life. We build three different machine learning models, named regression model, binary classification model and advanced multinomial classification model. And we also developed a novel data pre-processing procedure which is proved to be efficient. By applying models to three turbine failure data, we conclude that our advanced multinomial classification model is able to detect different types of component failures more than one month before they were detected by purpose-designed monitoring system. Our work develops a standard workflow which can be easily applied on all kinds of SCADA data. With low demand on additional sensors, our model gives very high prediction accuracy. It has potential to be applied in real O\&M work and help reduce its cost.

\end{abstract}